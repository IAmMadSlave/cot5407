%----------------------------------------------------------------------------------------
%	PACKAGES AND OTHER DOCUMENT CONFIGURATIONS
%----------------------------------------------------------------------------------------

\documentclass{article}

\usepackage{fancyhdr} % Required for custom headers
\usepackage{lastpage} % Required to determine the last page for the footer
\usepackage{extramarks} % Required for headers and footers
\usepackage{graphicx} % Required to insert images
\usepackage{listings}
\usepackage{color}
\usepackage{xcolor}
\usepackage{caption}
\usepackage{enumitem}
\usepackage{amsmath}
\usepackage{tikz}
\usepackage{mathtools}
\usetikzlibrary{calc,shapes.multipart,chains,arrows}

\DeclareCaptionFont{white}{\color{white}}
\DeclareCaptionFormat{listing}{%
\parbox{\textwidth}{\colorbox{gray}{\parbox{\textwidth}{#1#2#3}}\vskip-4pt}}
\captionsetup[lstlisting]{format=listing,labelfont=white,textfont=white}

% Margins
\topmargin=-0.45in
\evensidemargin=0in
\oddsidemargin=0in
\textwidth=6.5in
\textheight=9.0in
\headsep=0.25in 

\linespread{1.1} % Line spacing

% Set up the header and footer
\pagestyle{fancy}
\lhead{\hmwkAuthorName} % Top left header
\chead{\hmwkClass\ (\hmwkClassInstructor\ \hmwkClassTime): \hmwkTitle} % Top center header
\rhead{\hmwkDueDate} % Top right header
\lfoot{\lastxmark} % Bottom left footer
\cfoot{} % Bottom center footer
\rfoot{Page\ \thepage\ of\ \pageref{LastPage}} % Bottom right footer
\renewcommand\headrulewidth{0.4pt} % Size of the header rule
\renewcommand\footrulewidth{0.4pt} % Size of the footer rule

\setlength\parindent{0pt} % Removes all indentation from paragraphs

%----------------------------------------------------------------------------------------
%	DOCUMENT STRUCTURE COMMANDS
%	Skip this unless you know what you're doing
%----------------------------------------------------------------------------------------

\setcounter{secnumdepth}{0} % Removes default section numbers
\newcounter{homeworkProblemCounter} % Creates a counter to keep track of the number of problems

\newcommand{\homeworkProblemName}{}
\newenvironment{homeworkProblem}[1][Problem \arabic{homeworkProblemCounter}]{ % Makes a new environment called homeworkProblem which takes 1 argument (custom name) but the default is "Problem #"
\stepcounter{homeworkProblemCounter} % Increase counter for number of problems
\renewcommand{\homeworkProblemName}{#1} % Assign \homeworkProblemName the name of the problem
\section{\homeworkProblemName} % Make a section in the document with the custom problem count
}

%----------------------------------------------------------------------------------------
%   COLORS AND LANGUAGAGE
%----------------------------------------------------------------------------------------

\lstset{
    frame=lrb,xleftmargin=\fboxsep,xrightmargin=-\fboxsep,language=Java,basicstyle=\ttfamily,
    breaklines=true,columns=fullflexible,keepspaces=true,escapeinside={\%*}{*)}
       }

%----------------------------------------------------------------------------------------
%	NAME AND CLASS SECTION
%----------------------------------------------------------------------------------------

\newcommand{\hmwkTitle}{Homework\ \#3} % Assignment title
\newcommand{\hmwkDueDate}{Friday,\ Feburary\ 27,\ 2015} % Due date
\newcommand{\hmwkClass}{COT\ 5407} % Course/class
\newcommand{\hmwkClassTime}{5:00pm} % Class/lecture time
\newcommand{\hmwkClassInstructor}{Xie} % Teacher/lecturer
\newcommand{\hmwkAuthorName}{Musa V. Ahmed} % Your name

%----------------------------------------------------------------------------------------

%----------------------------------------------------------------------------------------
%   SYMBOLS AND STUFF
%----------------------------------------------------------------------------------------

%----------------------------------------------------------------------------------------

\begin{document}
\belowcaptionskip=-10pt
\DeclarePairedDelimiter\ceil{\lceil}{\rceil}
\DeclarePairedDelimiter\floor{\lfloor}{\rfloor}

%----------------------------------------------------------------------------------------
%	PROBLEM 1
%----------------------------------------------------------------------------------------

\begin{homeworkProblem}
    Give an algorithm that sorts $n$ $0/1$-valued records (that is, the key of
    each record is either 0 or 1). Your algorithm should be stable and run in
    $\mathcal{O}(n)$ time.

    \vspace{5 mm}
    \begin{lstlisting}[frame=none]
    # this is like a bucket sort
    # let A be list to be sorted
    list_zero = []
    list_one  = []
    for i=0 to A.length - 1:
        if A[i] == 0:
            list_zero.append( A[i] )
        else:
            list_one.append( A[i] )

    return list_zero + list_one
    \end{lstlisting}
        
\end{homeworkProblem}
\clearpage

%----------------------------------------------------------------------------------------

%----------------------------------------------------------------------------------------
%	PROBLEM 2
%----------------------------------------------------------------------------------------

\begin{homeworkProblem}
    Give an algorithm that sorts $n$ $0/1$-valued records. Your algorithm
    should be an in-place algorithm (that is, the algorithm uses a constant
    amount of storage space in addition to the original input array) and runs
    in $\mathcal{O}(n)$ time.

    \vspace{5 mm}
    \begin{lstlisting}[frame=none]
    # this is like a counting sort
    # let A be the array to be sorted
    counters = [0, 0]
    for i=0 to A.length-1:
        counters[A[i]] = counters[A[i]] + 1

    i = 0
    for j=0 to counter.length-1:
        for j=0 to counters.length-1:
            A[i] =  j
            i = i + 1

    return A
    \end{lstlisting}
\end{homeworkProblem}
\clearpage

%----------------------------------------------------------------------------------------

%----------------------------------------------------------------------------------------
%	PROBLEM 3
%----------------------------------------------------------------------------------------

\begin{homeworkProblem}
    Give an algorithm that sorts $n$ $0/1$-valued records. Your algorithm
    should be in-place and stable.

    \vspace{5 mm}
    \begin{lstlisting}[frame=none]
    # this is bubble sort
    # let A be the arry to be sorted
    for i=1 to A.length:
        for j=0 to A.length-1:
            if A[j] > A[j+1]:
                swap( A[j], A[j+1] )
    \end{lstlisting}
\end{homeworkProblem}
\clearpage

%----------------------------------------------------------------------------------------

%----------------------------------------------------------------------------------------
%	PROBLEM 4
%----------------------------------------------------------------------------------------

\begin{homeworkProblem}
    Can any of your algorithms from the first 3 questions be used to sort $n$
    records with $b$-bit keys using radix sort in $\mathcal{O}(bn)$ time?
    Explain how if yes or why if not.

    \vspace{5 mm}
    The algorithm in Problem 2 could be used. The algorithm provided in Problem
    1 is not stable. Finally, the algorithm provided in Problem 3 takes
    quadratic time.
\end{homeworkProblem}
\clearpage

%----------------------------------------------------------------------------------------

%----------------------------------------------------------------------------------------
%	PROBLEM 5
%----------------------------------------------------------------------------------------

\begin{homeworkProblem}
    Augment the binary search tree (BST) data structure so that it supports
    queries of the form "rank($k$)=?", where rank(k) is defined to be the
    number of nodes in the tree whose keys are less than or equal to $k$. The
    running time of each query should be $\mathcal{O}(h)$, where $h$ is the
    height of the BST. (that is, show how to add some additional fields to the
    BST data structure so that there is an efficient algorithm that can answer
    such queries.)

    \vspace{5 mm}
    The binary search tree can be augemented with an additional field for each
    node called $size$.
    This holds the number of nodes that are in the subtree of a node.
    \begin{lstlisting}[frame=none]
    rank( node ):
    r = size( left( node ) ) + 1
    temp = node
    while temp != root:
        if temp = right( parent( temp ) ):
            r = r + size( left( parent( temp ) ) ) + 1
        temp = parent( temp )

    return r
    \end{lstlisting}
\end{homeworkProblem}
\clearpage

%----------------------------------------------------------------------------------------

%----------------------------------------------------------------------------------------
%	PROBLEM 6
%----------------------------------------------------------------------------------------

\begin{homeworkProblem}
    Suppose there are $k$ processes $P_1, \hdots, P_k$, all competing for the
    exclusive access to a resource (e.g. a communication channel). Suppose we
    divide time into discrete \emph{rounds}. During each round, if there are at
    least two processes requestion the resource, then all the processes fail to
    get the resource. If there is a single process requests the resource at a
    round, then the request will be fulfilled. The main challenge is there is
    no communication among these $k$ processes. The purpose of this problem is
    to design and analyze a \emph{randomized} protocol for these processes so
    that after a reasonable number of rounds, all processes will get access to
    the resource (with high probility).\\

    The protocol is very simple: during each round, every process requests the
    resource with a uniform probability $p$ independently.

    \begin{enumerate}[label=(\alph*)]
        \item For each round and every fixed process, what is the probability
            that this process successfully accesses the resource? What value of
            $p$ maximizes this probability?
        \item Let the probability you just derived be $r$. Prove that $1/ek
            \leq r \leq 1/2k$.
        \item For a fixed process, what is the probability that is fails to
            access the resource for all the first $t$ rounds? How do you set
            the value of $t$ to make this failure probability less than
            $1/k^3$?
        \item Recall the very useful probability inequality of \emph{union
            bound} and give a value of $t$ such that the probability that
            there is a process that has not accessed the resource after
            $t$ rounds is at most $1/k^2$ (such a probability is extremely
            small when $k$ is larger).
    \end{enumerate}

    \vspace{3 mm}
    \begin{enumerate}[label=(\alph*)]
        \item Let $A[i,r]$ represent some process $P_i$ attempting to access the resource
            in round $r$.\\
            Then $Pr[A[i,r]] = p$ and $Pr[\overline{A[i,r]}] = 1-p$.\\
            Now let $S[i,r]$ represent some process $P_i$ successfully
            accessing the resource in round $r$.\\
            Then $S[i,r] = A[i,r] \cap (\cap_{i\neq j} \overline{A[j,r]})$ and
            now, \\
            $Pr[S[i,r]]=Pr[A[i,r]]*\prod_{i\neq
            j}Pr[\overline{A[j,r]}]=p(1-p)^{k-1}$\\

            The value of $p$ that maximizes this probability is $1/k$.
        \item $\lim_{k\to 2} (1-(1/k))^k = 1/4$ and $\lim_{k\to 2}
            (1-(1/k))^{(k-1)} = 1/2$.\\
            Thus, $1/ek \leq r \leq 1/2k$ is true.
        \item Let $F[i,r]$ represent some process $P_i$ failing to access the
            resource for all rounds $r$.\\
            Then $Pr[F[i,r]]=\cap^{r}_{i=1}\overline{S[i,r]}$ and then,\\
            $Pr[F[i,r]]=(1-Pr[S[i,r]])^r \leq (1- 1/ek)^r$.\\
            Now if we set $r=ek$ then,
            $Pr[F[i,r]]\leq(1-1/ek)^{\ceil{ek}}\leq(1-1/ek)^{ek}\leq1/e$.\\
            Also if we increase $r$ to $\ceil{ek}*c\ ln\ k$,\\
            $Pr[F[i,r]]\leq((1-1/ek)^{\ceil{ek}})^{c\ ln\ k}\leq(1/e)^{c\ ln\
            k}=e^{-c\ ln\ k}=k^{-c}$.
        \item Let $F[i,r]$ represent some process $P_i$ failing to access the
            resource for all rounds $r$.\\
            Then $Pr[F[i,r]]=\cap^{r}_{i=1}\overline{S[i,r]}$ and then,\\
            $Pr[F[i,r]]=(1-Pr[S[i,r]])^r \leq (1- 1/ek)^r$.\\
            Now if we set $r=ek$ then,
            $Pr[F[i,r]]\leq(1-1/ek)^{\ceil{ek}}\leq(1-1/ek)^{ek}\leq1/e$.\\
            Also if we increase $r$ to $\ceil{ek}*c\ ln\ k$,\\
            $Pr[F[i,r]]\leq((1-1/ek)^{\ceil{ek}})^{c\ ln\ k}\leq(1/e)^{c\ ln\
            k}=e^{-c\ ln\ k}=k^{-c}$.
    \end{enumerate}
\end{homeworkProblem}
\clearpage

%----------------------------------------------------------------------------------------


\end{document}
