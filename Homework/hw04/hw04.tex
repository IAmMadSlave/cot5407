%----------------------------------------------------------------------------------------
%	PACKAGES AND OTHER DOCUMENT CONFIGURATIONS
%----------------------------------------------------------------------------------------

\documentclass{article}

\usepackage{fancyhdr} % Required for custom headers
\usepackage{lastpage} % Required to determine the last page for the footer
\usepackage{extramarks} % Required for headers and footers
\usepackage{graphicx} % Required to insert images
\usepackage{listings}
\usepackage{color}
\usepackage{xcolor}
\usepackage{caption}
\usepackage{enumitem}
\usepackage{amsmath}
\usepackage{tikz}
\usepackage{mathtools}
\usetikzlibrary{calc,shapes.multipart,chains,arrows}

\DeclareCaptionFont{white}{\color{white}}
\DeclareCaptionFormat{listing}{%
\parbox{\textwidth}{\colorbox{gray}{\parbox{\textwidth}{#1#2#3}}\vskip-4pt}}
\captionsetup[lstlisting]{format=listing,labelfont=white,textfont=white}

% Margins
\topmargin=-0.45in
\evensidemargin=0in
\oddsidemargin=0in
\textwidth=6.5in
\textheight=9.0in
\headsep=0.25in 

\linespread{1.1} % Line spacing

% Set up the header and footer
\pagestyle{fancy}
\lhead{\hmwkAuthorName} % Top left header
\chead{\hmwkClass\ (\hmwkClassInstructor\ \hmwkClassTime): \hmwkTitle} % Top center header
\rhead{\hmwkDueDate} % Top right header
\lfoot{\lastxmark} % Bottom left footer
\cfoot{} % Bottom center footer
\rfoot{Page\ \thepage\ of\ \pageref{LastPage}} % Bottom right footer
\renewcommand\headrulewidth{0.4pt} % Size of the header rule
\renewcommand\footrulewidth{0.4pt} % Size of the footer rule

\setlength\parindent{0pt} % Removes all indentation from paragraphs

%----------------------------------------------------------------------------------------
%	DOCUMENT STRUCTURE COMMANDS
%	Skip this unless you know what you're doing
%----------------------------------------------------------------------------------------

\setcounter{secnumdepth}{0} % Removes default section numbers
\newcounter{homeworkProblemCounter} % Creates a counter to keep track of the number of problems

\newcommand{\homeworkProblemName}{}
\newenvironment{homeworkProblem}[1][Problem \arabic{homeworkProblemCounter}]{ % Makes a new environment called homeworkProblem which takes 1 argument (custom name) but the default is "Problem #"
\stepcounter{homeworkProblemCounter} % Increase counter for number of problems
\renewcommand{\homeworkProblemName}{#1} % Assign \homeworkProblemName the name of the problem
\section{\homeworkProblemName} % Make a section in the document with the custom problem count
}

%----------------------------------------------------------------------------------------
%   COLORS AND LANGUAGAGE
%----------------------------------------------------------------------------------------

\lstset{
    frame=lrb,xleftmargin=\fboxsep,xrightmargin=-\fboxsep,language=Java,basicstyle=\ttfamily,
    breaklines=true,columns=fullflexible,keepspaces=true,escapeinside={\%*}{*)}
       }

%----------------------------------------------------------------------------------------
%	NAME AND CLASS SECTION
%----------------------------------------------------------------------------------------

\newcommand{\hmwkTitle}{Homework\ \#4} % Assignment title
\newcommand{\hmwkDueDate}{Sunday,\ April\ 12,\ 2015} % Due date
\newcommand{\hmwkClass}{COT\ 5407} % Course/class
\newcommand{\hmwkClassTime}{5:00pm} % Class/lecture time
\newcommand{\hmwkClassInstructor}{Xie} % Teacher/lecturer
\newcommand{\hmwkAuthorName}{Musa V. Ahmed} % Your name

%----------------------------------------------------------------------------------------

%----------------------------------------------------------------------------------------
%   SYMBOLS AND STUFF
%----------------------------------------------------------------------------------------

%----------------------------------------------------------------------------------------

\begin{document}
\belowcaptionskip=-10pt
\DeclarePairedDelimiter\ceil{\lceil}{\rceil}
\DeclarePairedDelimiter\floor{\lfloor}{\rfloor}

%----------------------------------------------------------------------------------------
%	PROBLEM 1
%----------------------------------------------------------------------------------------

\begin{homeworkProblem}
    Consider a hash table of size $m$ that is used for storing $n$ items with
    $n \leq m/2$. Suppose open addressing is used to resolve collisions.

    \begin{enumerate}[label=(\alph*)]
        \item Assuming uniform hashing, show that for every $i = 1, 2, \hdots$,
            $n$, the probability that the $i$th insertion required $more$
            $than$ $k$ probes is at most $2^{-k}$, for $k = 1, 2, \hdots$
        \item Show that for every $i = 1, 2, \hdots$, $n$, the probability that
            $i$th insertion requires $more$ $than$ $2$ $log$ $n$ probes is at
            most $1/n^2$.
    \end{enumerate}


\end{homeworkProblem}
\clearpage

%----------------------------------------------------------------------------------------

%----------------------------------------------------------------------------------------
%	PROBLEM 2
%----------------------------------------------------------------------------------------

\begin{homeworkProblem}
    Define the $incidence$ $matrix$ $B$ of a directed graph with no self-loops
    to be an $n*m$ matrix with rows indexed by vertices, columns indexed by
    edges such that,

    \begin{equation}
        B_{i,j}=\begin{cases}
            -1, & \text{if edge $j$ leaves vertex $i$}, \\
            1, & \text{if edge $j$ enters vertex $i$}, \\
            0, & \text{otherwise}.
            \end{cases}
    \end{equation}

    Let $B^T$ be the transpose of matrix $B$. Find out what the entries of the
    $n*n$ matrix $BB^T$ stand for.

\end{homeworkProblem}
\clearpage

%----------------------------------------------------------------------------------------

%----------------------------------------------------------------------------------------
%	PROBLEM 3
%----------------------------------------------------------------------------------------

\begin{homeworkProblem}
    Given an undirected graph with $n$ vertices and $m$ edges, find an
    $\mathcal{O}{n+m}$ time algorithm that determines whether it is possible to
    color all the vertices red and blue such that every edge is between a red
    vertex and blue vertex. If such a coloring exists, your algorithm should
    produce one.

\end{homeworkProblem}
\clearpage

%----------------------------------------------------------------------------------------

%----------------------------------------------------------------------------------------
%	PROBLEM 4
%----------------------------------------------------------------------------------------

\begin{homeworkProblem}
    Given a grpah $G$ and a minimum spanning tree T, suppose we decrease the
    weight of one of the edges in $T$. Show that $T$ is still a minimum
    spanning tree for $G$.

\end{homeworkProblem}
\clearpage

%----------------------------------------------------------------------------------------

%----------------------------------------------------------------------------------------
%	PROBLEM 5
%----------------------------------------------------------------------------------------

\begin{homeworkProblem}
    Suppose that all edge weights in a graph are integers in the range from 1
    to $W$ for some constant $W$ (say, $W=20$). How fast can you make Prim's
    algorithm run? Express your answer in terms of $n, m, W$.

\end{homeworkProblem}
%\clearpage

%----------------------------------------------------------------------------------------


\end{document}
