%----------------------------------------------------------------------------------------
%	PACKAGES AND OTHER DOCUMENT CONFIGURATIONS
%----------------------------------------------------------------------------------------

\documentclass{article}

\usepackage{fancyhdr} % Required for custom headers
\usepackage{lastpage} % Required to determine the last page for the footer
\usepackage{extramarks} % Required for headers and footers
\usepackage{graphicx} % Required to insert images
\usepackage{listings}
\usepackage{color}
\usepackage{xcolor}
\usepackage{caption}
\usepackage{enumitem}
\usepackage{amsmath}
\usepackage{tikz}
\usepackage{mathtools}
\usetikzlibrary{calc,shapes.multipart,chains,arrows}

\DeclareCaptionFont{white}{\color{white}}
\DeclareCaptionFormat{listing}{%
\parbox{\textwidth}{\colorbox{gray}{\parbox{\textwidth}{#1#2#3}}\vskip-4pt}}
\captionsetup[lstlisting]{format=listing,labelfont=white,textfont=white}

% Margins
\topmargin=-0.45in
\evensidemargin=0in
\oddsidemargin=0in
\textwidth=6.5in
\textheight=9.0in
\headsep=0.25in 

\linespread{1.1} % Line spacing

% Set up the header and footer
\pagestyle{fancy}
\lhead{\hmwkAuthorName} % Top left header
\chead{\hmwkClass\ (\hmwkClassInstructor\ \hmwkClassTime): \hmwkTitle} % Top center header
\rhead{\hmwkDueDate} % Top right header
\lfoot{\lastxmark} % Bottom left footer
\cfoot{} % Bottom center footer
\rfoot{Page\ \thepage\ of\ \pageref{LastPage}} % Bottom right footer
\renewcommand\headrulewidth{0.4pt} % Size of the header rule
\renewcommand\footrulewidth{0.4pt} % Size of the footer rule

\setlength\parindent{0pt} % Removes all indentation from paragraphs

%----------------------------------------------------------------------------------------
%	DOCUMENT STRUCTURE COMMANDS
%	Skip this unless you know what you're doing
%----------------------------------------------------------------------------------------

\setcounter{secnumdepth}{0} % Removes default section numbers
\newcounter{homeworkProblemCounter} % Creates a counter to keep track of the number of problems

\newcommand{\homeworkProblemName}{}
\newenvironment{homeworkProblem}[1][Problem \arabic{homeworkProblemCounter}]{ % Makes a new environment called homeworkProblem which takes 1 argument (custom name) but the default is "Problem #"
\stepcounter{homeworkProblemCounter} % Increase counter for number of problems
\renewcommand{\homeworkProblemName}{#1} % Assign \homeworkProblemName the name of the problem
\section{\homeworkProblemName} % Make a section in the document with the custom problem count
}

%----------------------------------------------------------------------------------------
%   COLORS AND LANGUAGAGE
%----------------------------------------------------------------------------------------

\lstset{
    frame=lrb,xleftmargin=\fboxsep,xrightmargin=-\fboxsep,language=Java,basicstyle=\ttfamily,
    breaklines=true,columns=fullflexible,keepspaces=true,escapeinside={\%*}{*)}
       }

%----------------------------------------------------------------------------------------
%	NAME AND CLASS SECTION
%----------------------------------------------------------------------------------------

\newcommand{\hmwkTitle}{Homework\ \#2} % Assignment title
\newcommand{\hmwkDueDate}{Monday,\ Feburary\ 16,\ 2015} % Due date
\newcommand{\hmwkClass}{COT\ 5407} % Course/class
\newcommand{\hmwkClassTime}{5:00pm} % Class/lecture time
\newcommand{\hmwkClassInstructor}{Xie} % Teacher/lecturer
\newcommand{\hmwkAuthorName}{Musa V. Ahmed \\collaborated with Jose Acosta} % Your name

%----------------------------------------------------------------------------------------

%----------------------------------------------------------------------------------------
%   SYMBOLS AND STUFF
%----------------------------------------------------------------------------------------

%----------------------------------------------------------------------------------------

\begin{document}
\belowcaptionskip=-10pt
\DeclarePairedDelimiter\ceil{\lceil}{\rceil}
\DeclarePairedDelimiter\floor{\lfloor}{\rfloor}

%----------------------------------------------------------------------------------------
%	PROBLEM 1
%----------------------------------------------------------------------------------------

\begin{homeworkProblem}
    An array $A$ of \emph{n} distinct numbers are said to be \emph{unimodal} if there exists an
    index \emph{k}, $1 \leq k \leq n$, such that $A[1] \leq A[2] \leq \dots \leq A[k-1] \leq
    A[k]$ and $A[k] \geq A[k+1] \geq \dots \geq A[n]$. In other words $A$ has a unique peak and
    are monotone on both sides of the peak. Design an $\mathcal{O}$(log\ n) algorithm that finds the
    peak in a unimodal array of size \emph{n}.

    \vspace{5 mm}
    \begin{lstlisting}[frame=none]
    low = 0
    high = A.length // this is n

    while(1)
        i = (high - low) / 2
        if( A[i-1] < A[i] && A[i] < A[i+1] )
            low = i
        else if( A[i-1] > A[i] && A[i] > A[i+1] )
            high = i
        else
            return A[i]
    \end{lstlisting}

    \begin{center}
    $T(n)=T(n/2)+c$\\

    $T(n)=T(n/4)+c+c$\\

    $T(n)=T(n/2^k)+ck$\\

    if $k=log_2n$ \\

    $T(n)=T(n/{2^{log_2n}})+ c\ log_2n$\\

    $T(1) + c\ log_2n$\\

    $=\mathcal{O}$($log\ n$)
    \end{center}

\end{homeworkProblem}
\clearpage

%----------------------------------------------------------------------------------------

%----------------------------------------------------------------------------------------
%	PROBLEM 2
%----------------------------------------------------------------------------------------

\begin{homeworkProblem}
    Let $A$ be a heap of size \emph{n}. Let \emph{T} be the binary tree corresponding to $A$ (that
    is, $A[1]$ is the root node of $T$, $A[2]$ and $A[3]$ are the left child and right child
    nodes of $A[1]$, etc). Prove the following properties of $T$ (you need to pay attention
    to the floor notation in the equations).

    \begin{enumerate}[label=(\alph*)]
        \item The height of $T$ is $\floor*{log\ n}$; 
            
        The maximum number of nodes in a complete binary tree of height $h$ is,
        $2^{h+1}-1$\\

        The minimum number of nodes in a binary tree of height $h$ is, 
        $2^h$ this is because the maximum nodes in a binary tree of height $h-1$ is,
        $2^h-1$.\\

        Thus, $2^h \leq n < 2^{h+1}$\\

        Then by taking the log of all sides, $h \leq log(n) < h+1$.\\

        Since both $h$ and $h+1$ are integers, it follows thats $h= \floor*{log(n)}$\\

        \item $T$ is a \emph{balanced} binary tree, namely if we denote the two subtrees
        of the root node of $T$ by $T_L$ and $T_R$ respectively, then
        $|T_R| \leq |T_L| \leq 2n/3$ (In fact, you have just proved a more general
        statement: the inequality holds for any non-leaf node in the tree);\\

        The left child of the tree must fill up before the right child does as a result of
        being a balanced binary tree. The largest possible imbalance that can occur is the
        final row of the left subtree being completely populated and the final row of the right
        subtree being completely empty. The left subtree will be bounded by 2n/3.

        \item The leaf nodes of $T$ are $A[\floor*{n/2}+1]$, $A[\floor*{n/2}+2]$, \dots,
        $A[n]$.
        
        The last leaf is $n^{th}$ The parent node is $A[\floor*{n/2}]$. Thus leaf nodes are
        are from $A[\floor*{n/2}+1]$ to $n$.

    \end{enumerate}


\end{homeworkProblem}
\clearpage

%----------------------------------------------------------------------------------------

%----------------------------------------------------------------------------------------
%	PROBLEM 3
%----------------------------------------------------------------------------------------

\begin{homeworkProblem}
    Suppose you are given an unsorted list of \emph{n} distinct numbers. However, the list is
    close to sorted in the sense that each number is at most \emph{k} positions away from its
    position in the sorted list. For example, suppose the sorted list is in ascending
    order, then the smallest element in the original list will lie between position 1
    and position $k$ + 1, as position 1 is its position in the final sorted list.
    Design an efficient algorithm that sorts such a list. Your running time should
    depend on both \emph{n} and \emph{k}.

    \vspace{5 mm}
    \begin{lstlisting}[frame=none]
        A[] // unsorted list of distinct numbers
        H   // min-heap
        
        H = new Heap(k+1) // create min-heap of size k+1
        for( i=0, i<=k, i<n, i++)
            H[i] = A[i]
            H.heapify()

        // A will continue where it left off
        // j is the index for the min element in H
        for( i=k+1, j=0, j<n, i++, j++)
            if i<n
                A[j] = H.replaceMinimum(A[i])
            else
                A[j] = H.removeMinimum()
    \end{lstlisting}

    \begin{center}
        Make heap of size $k$ will take $\mathcal{O}$($k$) time. Remove the minimum element
        from the heap and add it to the array then from the left over elements add one to the
        heap.

        Thus, $\mathcal{O}(k)+\mathcal{O}((n-k)*log\ k)$
     \end{center}
\end{homeworkProblem}
\clearpage

%----------------------------------------------------------------------------------------

%----------------------------------------------------------------------------------------
%	PROBLEM 4
%----------------------------------------------------------------------------------------

\begin{homeworkProblem}
    Using the approach of Divide-and-Conquer, design an algorithm that finds the \emph{median}
    of an unsorted list (the \emph{median} of a list of \emph{n} number is the $\ceil*{n/2}$-th
    smallest number in the list; that is, the one that lies in the middle). State both the
    worst-case running time and the average-case running time of your algorithm. You do not
    need to give rigorous mathematical proofs of your claims but you should provide reasoning
    to support your conclusions. (hint: you will find the approach of \textbf{QuickSort} useful
    for this problem as well. Also, when applying the method of recursion, you will often find
    that working with a more \emph{general} problem is a lot easier.)

    \vspace{5 mm}
    In order to find the k-th smallest element of an unsorted list (k being the median) an
    algorithm similar to \textbf{QuickSort} is used. However, the array only need be partially
    sorted as we will know which side to work on through comparing the pivot element to the
    k-th element.

    \begin{lstlisting}[frame=none]
        k = ceil(list.length / 2) 
        select (list, start, end, k)
            if start==end
                return list[start]
            pivotIndex = [random number between start and end]
            pivotIndex = partition(list, start, end, pivotIndex)

            // after partition the pivot is in its final position
            if k == pivotIndex
                return list[k]
            else if k < pivotIndex // median is below the pivot
                return select( list, start, pivotIndex-1, k)
            else
                return select( list, pivotIndex+1, end, k)
    \end{lstlisting}

    Worst Case: $\mathcal{O}(n^2)$ \\
    If bad pivots are chosen each time then it could be possible that no work will be done and
    the list will be sorted in a worst case QuickSort style.\\

    Average Case: $\mathcal{O}(n)$ \\
    If good pivots are chosen (those which decrease problem size) then the search set decreases
    in size exponentially. The running time looks like $T(n) = T(n/b) + \mathcal{O}(n)$, however, since
    the work is divided by some fraction we will only be working on one side. Here Case 1 of
    the Master Theorem always applies since $log_b 1=0$ and $c=1$ meaning $c>log_b a$ in other
    words the running time is $\mathcal{O}(n)$
\end{homeworkProblem}
\clearpage

%----------------------------------------------------------------------------------------

%----------------------------------------------------------------------------------------
%	PROBLEM 5
%----------------------------------------------------------------------------------------

\begin{homeworkProblem}
    You have \emph{n} pairs of nuts and bolts (the bolts screw into the nuts) that have been
    dropped onto a table with the nuts on one side and the bolts on another side; however you
    cannot tell whether one nut is bigger than another nut or whether one bolt is bigger than
    another bolt. You can attempt to screw a bolt into a nut and this will tell you either that
    the nut matches the bolt, or that the bolt is too large or too small for the nut. Call this
    a \emph{nut/bolt comparison}. Give a randomized algorithm that will match all nuts with
    their corresponding bolts with at most $\mathcal{O}$(n\ log\ n) nut/bolt comparisons.

    \vspace{5 mm}
    This problem requires performing QuickSort on the nuts array and the bolts array while
    using the nuts as pivots for the bolts and vice versa. Select a random nut from the array and 
    treat it as the pivot for the bolts. Partition the bolts into those are larger than the nut and
    those that are smaller. During the process of partitioning the bolts the matching bolt for the pivot
    nut will be found. Place the bolt in the same index as the pivot nut. After the partitioning is complete
    use the bolt that matches the pivot nut to partition the nuts. This will result a bolt/nut pair, a set of
    unpaired nuts and bolts which are smaller than the pair used in partitioning, and a set of unpaired nuts
    and bolts which are larger than the pair used in partitioning.  Next recursively sort the two sets with 
    the same algorithms which will result in all nuts and bolts being sorted.
    
    Since we are using a randomized version of \textbf{QuickSort}  the worst case
    $\mathcal{O}(n^2)$ is extremely unlikely to show up and you will deal with at most 
    $\mathcal{O}(n\ log\ n)$ comparisons to find the corresponding pairs of nuts and 
    bolts(which will also be sorted).
\end{homeworkProblem}
%\clearpage

%----------------------------------------------------------------------------------------


\end{document}
