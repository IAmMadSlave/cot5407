%----------------------------------------------------------------------------------------
%	PACKAGES AND OTHER DOCUMENT CONFIGURATIONS
%----------------------------------------------------------------------------------------

\documentclass{article}

\usepackage{fancyhdr} % Required for custom headers
\usepackage{lastpage} % Required to determine the last page for the footer
\usepackage{extramarks} % Required for headers and footers
\usepackage{graphicx} % Required to insert images
\usepackage{listings}
\usepackage{color}
\usepackage{xcolor}
\usepackage{caption}
\usepackage{enumitem}
\usepackage{amsmath}
\usepackage{tikz}
\usetikzlibrary{calc,shapes.multipart,chains,arrows}

\DeclareCaptionFont{white}{\color{white}}
\DeclareCaptionFormat{listing}{%
\parbox{\textwidth}{\colorbox{gray}{\parbox{\textwidth}{#1#2#3}}\vskip-4pt}}
\captionsetup[lstlisting]{format=listing,labelfont=white,textfont=white}

% Margins
\topmargin=-0.45in
\evensidemargin=0in
\oddsidemargin=0in
\textwidth=6.5in
\textheight=9.0in
\headsep=0.25in 

\linespread{1.1} % Line spacing

% Set up the header and footer
\pagestyle{fancy}
\lhead{\hmwkAuthorName} % Top left header
\chead{\hmwkClass\ (\hmwkClassInstructor\ \hmwkClassTime): \hmwkTitle} % Top center header
\rhead{\hmwkDueDate} % Top right header
\lfoot{\lastxmark} % Bottom left footer
\cfoot{} % Bottom center footer
\rfoot{Page\ \thepage\ of\ \pageref{LastPage}} % Bottom right footer
\renewcommand\headrulewidth{0.4pt} % Size of the header rule
\renewcommand\footrulewidth{0.4pt} % Size of the footer rule

\setlength\parindent{0pt} % Removes all indentation from paragraphs

%----------------------------------------------------------------------------------------
%	DOCUMENT STRUCTURE COMMANDS
%	Skip this unless you know what you're doing
%----------------------------------------------------------------------------------------

\setcounter{secnumdepth}{0} % Removes default section numbers
\newcounter{homeworkProblemCounter} % Creates a counter to keep track of the number of problems

\newcommand{\homeworkProblemName}{}
\newenvironment{homeworkProblem}[1][Problem \arabic{homeworkProblemCounter}]{ % Makes a new environment called homeworkProblem which takes 1 argument (custom name) but the default is "Problem #"
\stepcounter{homeworkProblemCounter} % Increase counter for number of problems
\renewcommand{\homeworkProblemName}{#1} % Assign \homeworkProblemName the name of the problem
\section{\homeworkProblemName} % Make a section in the document with the custom problem count
}

%----------------------------------------------------------------------------------------
%   COLORS AND LANGUAGAGE
%----------------------------------------------------------------------------------------

\lstset{
    frame=lrb,xleftmargin=\fboxsep,xrightmargin=-\fboxsep,language=Java,basicstyle=\ttfamily,
    breaklines=true,columns=fullflexible,keepspaces=true,escapeinside={\%*}{*)}
       }

%----------------------------------------------------------------------------------------
%	NAME AND CLASS SECTION
%----------------------------------------------------------------------------------------

\newcommand{\hmwkTitle}{Homework\ \#1} % Assignment title
\newcommand{\hmwkDueDate}{Friday,\ January\ 30,\ 2015} % Due date
\newcommand{\hmwkClass}{COT\ 5407} % Course/class
\newcommand{\hmwkClassTime}{5:00pm} % Class/lecture time
\newcommand{\hmwkClassInstructor}{Xie} % Teacher/lecturer
\newcommand{\hmwkAuthorName}{Musa V. Ahmed \\collaborated with Jose Acosta} % Your name

%----------------------------------------------------------------------------------------

\begin{document}
\belowcaptionskip=-10pt

%----------------------------------------------------------------------------------------
%	PROBLEM 1
%----------------------------------------------------------------------------------------

\begin{homeworkProblem}
    Order the following function by growth rate. $n!$, $n^2+\sqrt{n}$ $log^{10}n$, $n^{1/3}$,
    $log^{100}n$, $n^3$, $2^n$, $10^{\sqrt{n}}$, $2^{log n}$, $2^{2 log n}$, $2^{\sqrt{log
    n}}$, $128$, $128n$. Indicate which functions at the same rate (all
    logarithms are base 2). For example, if you are asked to order $n$, $2n$, $2n^2$, then your answer should be
    "$n=\Theta{(2n)}, 2n=o(2n^2)$".

    \begin{enumerate}
        \item $128$ 
            $$\lim_{n\to\infty} (\frac{128}{log^{100}n}) = 0,\ 128=\mathcal{O}(log^{100}n)$$
        \item $log^{100}n$ 
            $$\lim_{n\to\infty} (\frac{log^{100}n}{2^{\sqrt{log n}}}) = 0, \ log^{100}n=\mathcal{O}(2^{\sqrt{log n}})$$
        \item $2^{\sqrt{log n}}$ 
            $$\lim_{n\to\infty} (\frac{2^{\sqrt{log n}}}{n^{1/3}}) = 0, \ 2^{\sqrt{log n}}=\mathcal{O}(n^{1/3})$$
        \item $n^{1/3}$ 
            $$\lim_{n\to\infty} (\frac{n^{1/3}}{2^{log n}}) = 0, \ n^{1/3}=\mathcal{O}(2^{log n})$$
        \item $2^{log n}$ 
            $$\lim_{n\to\infty} (\frac{2^{log n}}{128n})=0, \ 2^{log n}=\mathcal{O}(128n)$$
        \item $128n$ 
            $$\lim_{n\to\infty} (\frac{128n}{2^{2 log n}})=0, \ 128n=\mathcal{O}(2^{2 log n})$$
       \item $2^{2 log n}$ 
           $$\lim_{n\to\infty} (\frac{2^{2 log n}}{n^2+\sqrt{n} log^{10}n})=0, \ 2^{2 log
           n}=\mathcal{O}(n^2+\sqrt{n} log^{10}n)$$
        \item $n^2+\sqrt{n} log^{10}n$ 
            $$\lim_{n\to\infty} (\frac{n^2+\sqrt{n} log^{10}n}{n^3})=0, \ n^2+\sqrt{n} log^{10}n=\mathcal{O}(n^3))$$
        \item $n^3$ 
            $$\lim_{n\to\infty} (\frac{n^3}{10^{\sqrt{n}}})=0, \ n^3=\mathcal{O}(10^{\sqrt{n}})$$
        \item $10^{\sqrt{n}}$ 
            $$\lim_{n\to\infty} (\frac{10^{\sqrt{n}}}{2^n})=0, \ =10^{\sqrt{n}}\mathcal{O}(2^n)$$
        \item $2^n$
            $$\lim_{n\to\infty} (\frac{2^n}{n!})=0, \ 2^n=\mathcal{O}(n!)$$
        \item $n!$
            $$\lim_{n\to\infty} (\frac{n!}{2^n})=\infty, \ n!=o(2^n)$$
    \end{enumerate}
\end{homeworkProblem}
\clearpage

%----------------------------------------------------------------------------------------

%----------------------------------------------------------------------------------------
%	PROBLEM 2
%----------------------------------------------------------------------------------------

\begin{homeworkProblem}
    Solve the following recurrence equations, expressing the answer in Big-Oh notation. Assume
    that $T{n}$ is constant for sufficiently small $n$.

    \begin{enumerate}[label=(\alph*)]
        \item $T(n) = T(n/2) + 100$
            $$a=1,\  b=1,\  c=0,\ f(n)=100$$
            $$f(n)=\Theta(n^clog^kn)\ when\ c=0, k=0$$
            $$log_b a=log_2 1=0\ c=log_b a$$
            \begin{center}
                This is case 2.
            \end{center}
            $$T(n)=\Theta(n^{log_b a} log^{k+1}n)=\Theta(n^0 log^1 n)=\Theta(log \ n)$$
        \item $T(n) = 8T(n/2) + n^2$ 
            $$a=8,\ b=2,\ c=2,\ f(n)=n^2$$
            $$log_b a=log_2 8=3\ c<log_b a$$
            \begin{center}
                This is case 1.
            \end{center}
            $$T(n)=\Theta(n^{log_b a})=\Theta(n^3)$$
        \item $T(n) = 8T(n/2) + n^3$
            $$a=8,\ b=2,\ c=3,\ f(n)=n^2$$
            $$f(n)=\Theta(n^c log^k n) \ c=3,\ k=0$$
            $$log_b a=log_2 8=3\ c=log_b a$$
            \begin{center}
                This is case 2.
            \end{center}
            $$T(n)=\Theta(n^{log_b a}log^{k+1}n)=\Theta(n^3 log \ n)$$
        \item $T(n) = 8T(n/2) + n^4$
            $$a=8,\ b=2,\ c=4,\ f(n)=n^4$$
            $$f(n)=\Omega(n^c)\ when\ c=4$$
            $$log_b a=log_2 8=3\ c>log_b a$$
            $$af(\frac{n}{b})\leq kf(n),\ where k<1$$
            $$8(\frac{n^4}{16})<kn^4$$
            $$\frac{n^4}{2} \leq kn^4$$
            \begin{center}
                This is case 3.
            \end{center}
            $$T(n)=\Theta(f(n))=\Theta(n^4)$$
        \item $T(n) = T(n-1) + log n$
            $$T(n-1) = T(n-2) + log(n-1) + log\ n$$
            $$\vdots$$
            $$T(n) = T(n-k) + log\ 1 + log\ 2 + \hdots + log(n-1) + log\ n$$
            $$Since\, log\ n! = \Sigma{log\ n}$$
            $$T(n) = T(n-k) + log\ n!$$
            $$=\Theta(log\ n!)$$
        \item $T(n) = T(n-3) + n$
            $$T(n-1) = T(n-2) + log(n-1) + log\ n$$
            $$\vdots$$
            $$T(n) = T(n-k) + log\ 1 + log\ 2 + \hdots + log(n-1) + log\ n$$
            $$Since\, log\ n! = \Sigma{log\ n}$$
            $$T(n) = T(n-k) + log\ n!$$
            $$=\Theta(log\ n!)$$
    \end{enumerate}
\end{homeworkProblem}
\clearpage

%----------------------------------------------------------------------------------------

%----------------------------------------------------------------------------------------
%	PROBLEM 3
%----------------------------------------------------------------------------------------

\begin{homeworkProblem}
   You implemented a quadratic time algorithm for a problem $P$. On a test run, your algorithm
   takes 50 seconds for inputs of size 1000. Your classmate found a clever algorithm solving
   the same problem with a running time $O(n^{3/2})$. However, the faster algorithm takes 150
   seconds for input of size 1000. Explain how can this happen. If you need to solve a problem
   of size 4000, which algorithm you should use? What about input of size 10,000? Explain your
   answers (assume low-order terms are negligible).

   \vspace{10 mm}
   \begin{center}
        \includegraphics[scale=0.75]{prob3graph}
    \end{center}

    Looking at the graph the $\mathcal{O}(n^{3/2})$ function (shown in blue) initial grows at a
much faster rate than the $\mathcal{O}(n^2)$ function (shown in red) hence why the "faster"
algorithm takes longer for input size 1000. \\
    For input 4,000 either algorithm could be used since input size 4,000 is still relatively
    small. However, for input size 10,000 the more clever algorithm might be more appropriate
    since according to the graph as the input size increases the growth slows in comparison to
    the other.
\end{homeworkProblem}
\clearpage

%----------------------------------------------------------------------------------------

%----------------------------------------------------------------------------------------
%	PROBLEM 4
%----------------------------------------------------------------------------------------

\begin{homeworkProblem}
    Recall that in the \emph{testing safe height to drop a cellphone} problem we discussed in
    the class, the goal is to find out the maximum safe height to drop a cellphone without
    breaking it. In the class we saw that if the maximum safe height is $n$, then in the worst
    case we can perform $n$ tests if there is only one cellphone and $2\sqrt{n}$ tests if there
    are two cellphones. Give an algorithm to minimize the number of tests if there are $k$
    cellphones available (assume $k$ is constant). How many tests do you need to perform?

    \vspace{10 mm}
    Starting at zero increase the drop height exponentially until a cellphone breaks.\\
    Once a cellphone has broken perform a binary search between the last interval and the
    interval in which the last phone broke. \\

    The number of tests needed for this algorithm would be $log\ n + log\ m$, where $m$ is the
    size in between the last two exponential intervals that contain the maximum safe height to
    drop.
\end{homeworkProblem}
\clearpage

%----------------------------------------------------------------------------------------

%----------------------------------------------------------------------------------------
%	PROBLEM 5
%----------------------------------------------------------------------------------------

\begin{homeworkProblem}
    You are given a set of $n$ numbers. Give an $O(n^2)$ algorithm to decide if there exist
    three numbers $a$, $b$, and $c$ in the set such that $a + b = c$ (Hint: sort the numbers
    first). 

    \vspace{10 mm}
    \begin{lstlisting}[frame=none]
        int arr = array[n];

        sort(arr); 
    
        for(int i = 0; i < n; i++) {
            int j = i;
            int k =i++;

            while(j < n && k < n) {
                if( ( (a[k] - a[j]) > a[i] {
                    j++;
                }
                else if( (a[k] - a[j]) < a[i]) {
                    k++;
                }
                else {
                    return true;
                }
            }
        }
        
        return false;
    \end{lstlisting}

\end{homeworkProblem}
%\clearpage

%----------------------------------------------------------------------------------------


\end{document}
